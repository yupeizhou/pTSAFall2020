\documentclass[12pt]{article}

\usepackage[a4paper,margin=0.5in]{geometry}

\usepackage[square,numbers,sort&compress]{natbib}
%\usepackage[sort&compress]{natbib}

\usepackage[utf8]{inputenc} % allow utf-8 input
\usepackage[T1]{fontenc}    % use 8-bit T1 fonts
\usepackage{hyperref}       % hyperlinks
\usepackage{url}            % simple URL typesetting
\usepackage{booktabs}       % professional-quality tables
\usepackage{amsfonts}       % blackboard math symbols
\usepackage{nicefrac}       % compact symbols for 1/2, etc.
\usepackage{microtype}      % microtypography
\usepackage{amsmath}
\usepackage{algorithm}
\usepackage[noend]{algpseudocode}

\usepackage{graphicx}
\newcommand{\bigo}[1]{{\cal O}\left(#1 \right)}
\newcommand{\p}[1]{\mathrm{P}\left(#1 \right)}
\newcommand{\vect}[1]{\mathbf{#1}}
\newcommand{\tr}{^\mathrm{t}}

\begin{document}
\thispagestyle{empty}
\begin{center}

\textbf{DS-GA 3001.001 Special Topics in Data Science: Probabilistic Time Series Analysis\\
Homework 3}
\end{center}

\noindent \textbf{Due date: November 6th, by 6pm}\\


\noindent \textbf{Problem 1.} (15p)\\
Consider the HMM with K=3 latent states and discrete observations $\{1,2,3\}$, with parameters specified by:
initial distribution $\pi = [1, \, 0,\, 0]$, 
transition matrix 
$ \mathbf{A} = \begin{bmatrix}
    0 & 0.5 & 0.5\\
    1 & 0 & 0 \\
    0 & 1 & 0
    \end{bmatrix}
$, where $A_{ij} = \mathrm{P}(z_{t+1}= j | z_t= i)$ 
and likelihood $\mathrm{P}(x_t|z_t)$ described by matrix entries $B_{xz}$:
$ \mathbf{B} =  \begin{bmatrix}
    0.5 & 0.5 & 0 \\
    0.5 & 0 & 0.5 \\
    0 & 0.5 & 0.5
    \end{bmatrix}.
    $\\
 Write down all possible state sequences consistent with observations a) 1, 2, 3 and b) 1, 3, 1.\\

\noindent \textbf{Problem 2.} (15p)\\
Construct an HMM that generates the observation sequence $A^{k_1}C^{k_2}A^{k_3}C^{k_4}$ where $A^{k_1}$ denotes $k_1$ repeats of symbol $A$ and the number of repeats $k_i$ are drawn from the set $\{1,2,3\}$ with equal probability.\\

\noindent \textbf{Problem 3.}  (20p)\\ 
Implement EM for an HMM model with K states and gaussian observations (full derivations in handout). 
Use this code to fit the weekly S\&P 500 returns data (data/sp500w.csv) for K = 2 vs. K = 3 and compare the two results. \\
Hint: You can reuse some of the inference code you've worked out for the lab. 
Use Example 6.17 from tsa4.pdf (yellow textbook) as guideline for plots and interpretation.

\end{document}
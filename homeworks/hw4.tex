\documentclass[12pt]{article}

\usepackage[a4paper,margin=0.5in]{geometry}

\usepackage[square,numbers,sort&compress]{natbib}
%\usepackage[sort&compress]{natbib}

\usepackage[utf8]{inputenc} % allow utf-8 input
\usepackage[T1]{fontenc}    % use 8-bit T1 fonts
\usepackage{hyperref}       % hyperlinks
\usepackage{url}            % simple URL typesetting
\usepackage{booktabs}       % professional-quality tables
\usepackage{amsfonts}       % blackboard math symbols
\usepackage{nicefrac}       % compact symbols for 1/2, etc.
\usepackage{microtype}      % microtypography
\usepackage{amsmath}
\usepackage{algorithm}
\usepackage[noend]{algpseudocode}

\usepackage{graphicx}
\newcommand{\bigo}[1]{{\cal O}\left(#1 \right)}
\newcommand{\p}{\mathrm{P}}
\newcommand{\vect}[1]{\mathbf{#1}}
\newcommand{\Var}{\mathrm{Var}}
\newcommand{\tr}{^\top}
\begin{document}
\thispagestyle{empty}
\begin{center}

\textbf{DS-GA 3001.001 Probabilistic Time Series Analysis\\
Homework 4}
\end{center}

\noindent \textbf{Due date: Nov 27, 8pm}\\
\\
\noindent \textbf{Problem 1. (15pt)} Which of these objects are a Gaussian process?
\begin{itemize}
\item linear combination of 2 GPs: $f(x) = a f_1(x) + b f_2(x)$ where $f_i \sim \mathcal{GP}(\mu_i(x); k_i(x,y))$ (independent) and $a,\, b$ are fixed parameters.
\item random linear: $ f(x) = a x + w$ where $a \sim \mathcal{N}(0,\sigma_a^2)$, $w\sim \mathcal{N}(0,\sigma_w^2)$.
\item random periodic: $f(x) = a \cos(wx)+ b \sin(wx)$ with $a \sim \mathcal{N}(0,\sigma^2)$, $b \sim \mathcal{N}(0,\sigma^2)$, w fixed parameter.
\end{itemize}
If yes, then write down the corresponding mean and covariance functions.\\

\noindent \textbf{Problem 2. (10pt)} 
How would you construct a GP-equivalent of an ARIMA (1,2,1) model (same type of qualitative features in the first two moments of data statistics, but in continuos time) ?\\

\noindent \textbf{Problem 3. (5pt)} 
What GP-based model would you use for the Johnson\&Johnson quarterly earnings database?
Explain your choices. Would it matter if the goal of your analysis is to interpolate to account for missing data in the middle of the recorded time interval vs.\ extrapolating a decade into the future?\\

\noindent \textbf{Optional (extra credit, 10pt)}
Fit your choice of nonlinear GP regression model on the data using the function in the period 1970-1974 and 1980-1990 (extrapolation in the future) as test points and the rest of the dataset for training. Compare your predictions against the actual data and comment on result.\\

\end{document}